\documentclass[12pt,reqno]{amsart}
\usepackage{geometry}
\geometry{left=25mm,right=25mm, top=23mm, bottom=25mm}

\setcounter{tocdepth}{4}
\setcounter{secnumdepth}{4}

\usepackage{polyglossia}
\setdefaultlanguage{portuguese}
\usepackage{textcomp}
\usepackage[T1]{fontenc}
\usepackage{lmodern}
\usepackage{fullpage}
\usepackage{psfrag}
\usepackage{srcltx}

\usepackage[hidelinks,
            colorlinks=true,
            linkcolor=magenta]{hyperref}
\usepackage{fontspec}
\setmonofont[]{Iosevka}
\usepackage{listings}
\lstset{basicstyle=\ttfamily,breaklines=true,numbers=left}
\usepackage{pgfplots, pgfplotstable, booktabs}
\pgfplotsset{compat=1.5}
\usepackage{siunitx}
\DeclareSIUnit{\sample}{sample}

\makeatletter
\renewcommand{\labelenumi}{\theenumi.}
\makeatother

%%%%%%%%%%%%%%%%%%%%%%%%%%%%%%%%%%%%%%%%%%%%%%%%%%%%%%%%%%%%%%%%%%%%%%
\let\eps\varepsilon
%%%%%%%%%%%%%%%%%%%%%%%%%%%%%%%%%%%%%%%%%%%%%%%%%%%%%%%%%%%%%%%%%%%%%%

\usepackage{setspace}
\onehalfspacing
\renewcommand{\baselinestretch}{1.2}

\title[]{Análise da eficiência do algoritmo de merge sort}
\author{Isabella Basso do Amaral --- 11810773}

\thanks{isabellabdoamaral@usp.br}


\begin{document}

\maketitle
\thispagestyle{empty}


%\indent{\bf Aluna:} Isabella Basso do Amaral\\
%\indent{\bf Orientador:} Guilherme Oliveira Mota


\section{Introdução}
Esta é uma análise de eficiência simplificada, utilizando o método
de \textit{doubling}\footnote{Aviso:
A implementação do algoritmo, gerador e scripts utilizados para análise
de dados foram feitos pela autora e não serão discutidos em profundidade
no presente trabalho.
}, conforme fora estudado na disciplina CCM0128, do curso de Ciências
Moleculares da USP.

\section{Método}
\subsection{\textit{doubling}}
Na presente análise utilizaremos o método de \textit{doubling}, o qual
nos dá uma estimativa da complexidade
do algoritmo estudado através do processo de dobrar a quantidade de
entradas a cada iteração, de tal forma que facilita a visualização
da taxa de crescimento da complexidade algorítmica do que estamos
analisando. I.e. para inputs sucessivos de $1, 2$ e $4$, para uma
algoritmo com complexidade de ordem $\mathcal{O}(n)$ esperamos
um resultado com tempos $\alpha, 2\alpha$ e $4\alpha$, respectivamente,
enquanto que para um algoritmo com complexidade de ordem $\mathcal{O}(n^2)$
esperamos tempos de $\alpha, 4\alpha$ e $16\alpha$.

\subsection{Medidas}
Para que não haja \textit{overhead} da função de geração (veja \ref{code:python})
nos tempos medidos, criaremos os arquivos a serem organizados de antemão.

Em cada passo do \textit{doubling} realizaremos 100 medições, tirando sua
média ao final (veja \ref{code:bash}).

\section{Resultados}
Podemos ver que o script gerou os seguintes dados no arquivo \texttt{.csv}.

\begin{table}[h!]
  \centering
  \caption{\textit{runtime} do \texttt{merge sort}}
  \pgfplotstabletypeset[ multicolumn names=l, % allows to have multicolumn names
  col sep=comma, % the seperator in our .csv file
    display columns/0/.style={ column name={sample size}, % name of first column
      column type={S},
      string type}, % use siunitx for formatting
    display columns/1/.style={ column name={avg time},
      column type={S},
      string type},
    every head row/.style={
      before row={\toprule}, % have a rule at top
      after row={\midrule} % rule under units
    },
    every last row/.style={after row=\bottomrule}, % rule at bottom
    ]{results.csv} % filename/path to file
\end{table}

\begin{center}
  \begin{tikzpicture}
    \begin{axis}[
      title={\textit{doubling method (runtime)}},
      xlabel={amostra},
      ylabel={tempo médio para 100 iterações [s]}
    ]
      \addplot table [x=sample size, y=avg time, col sep=comma] {results.csv};
    \end{axis}
  \end{tikzpicture}
\end{center}

O que se aproxima de uma reta com coeficiente angular aproximado
\[\alpha=\dfrac{1.398-0.195}{(1.28-0.01)\cdot10^6}\approx\SI[per-mode=symbol]{0.95}{\nano\second\per\sample} \]

\section{Discussão}
\subsection{Complexidade}
Dado que o \texttt{runtime} do algoritmo aproxima uma reta, podemos inferir que
sua complexidade é $\mathcal{O}(n)$.

\subsection{Estabilidade}
Observando a função \texttt{merge} (linhas 19--44 do arquivo \ref{code:java}),
nota-se que, enquanto o índice \texttt{k} aumenta, o índice \texttt{i}
que acompanha o array inferior é comparado ao índice \texttt{j}
que acompanha o array superior e, dessa forma, para garantir
estabilidade é necessário que preservemos essa relação.

No presente trecho (retirado do mesmo arquivo) notamos que, conforme desejado,
o array inferior é favorecido em casos de igualdade e, conforme iteramos com
o índice \texttt{k} aumentando, a implementação é estável.

\begin{lstlisting}[firstnumber=37]
            if (lower[i].compareTo(upper[j]) <= 0)
                merger[k] = lower[i++];
            else
                merger[k] = upper[j++];
\end{lstlisting}


\clearpage
\section{Apêndices}

\subsection{Apêndice A}
\label{code:java}

Arquivo de \texttt{merge sort} (em Java, reescrito por simplicidade).

\lstset{language=Java}
\lstinputlisting{minimalMergeSort.java}

\clearpage
\subsection{Apêndice B}
\label{code:python}

Arquivo gerador de strings aleatórias (em Python).

\lstset{language=Python}
\lstinputlisting{str_gen}

\clearpage
\subsection{Apêndice C}
\label{code:python2}

Arquivo de média (em Python).

\lstinputlisting{get_averaged_table}

\clearpage
\subsection{Apêndice D}
\label{code:bash}

Script de \textit{timing}.

\lstset{language=Bash}
\lstinputlisting{time.sh}

\end{document}
