\documentclass{article}

\usepackage{IMTtikz}

\title{Um breve resultado sobre partições}
\author{Isabella B.}
\newtheorem{theorem}{Teorema}
\newtheorem{corollary}{Corollary}
\newtheorem{lemma}{Lemma}
\let\oldemptyset\emptyset
\let\emptyset\varnothing
\DeclareMathOperator{\ex}{ex}
\DeclarePairedDelimiter\ceil{\lceil}{\rceil}
\makeatletter
\let\oldceil\ceil
\def\ceil{\@ifstar{\oldceil}{\oldceil*}}
\makeatother
\DeclarePairedDelimiter\floor{\lfloor}{\rfloor}
\makeatletter
\let\oldfloor\floor
\def\floor{\@ifstar{\oldfloor}{\oldfloor*}}
\makeatother

\begin{document}
    \maketitle
    
    Seja $N$ um conjunto que deve ser particionado em parcelas com, no máximo, $M$ objetos.
    Denotamos a quantidade de partições atendendo esses requisitos por $p(N, M)$.
    
    \begin{theorem}
        \[ p(N, M) = p(N - M, M) + p(N, M - 1) \]
    \end{theorem}
    \begin{proof}
        Podemos notar que, para qualquer $N\in\mathbb{N}$ temos $p(N, 1) = 1$,
        já que só existe uma soma única de $N$ parcelas de $1$.
        Da mesma forma, é possível notar que para qualquer $N\in\mathbb{N}$,
        temos $p(N, 2)=\ceil{N/2}$.
        
        Tomando nosso caso inicial de $M=2$:
        \[ p(N, 2) = p(N - 2, 2) + p(N, 1) \rightarrow p(N, 2) - p(N - 2, 2) = \ceil{N/2} - \ceil{(N - 2)/2} = 1 = p(N, 1), \]
        o que é claramente verdadeiro.

        Portanto, por indução, temos que:
        \[ p(N, k + 1) - p(N - (k + 1), k + 1) = p(N, k) \]


    \end{proof}
\end{document}